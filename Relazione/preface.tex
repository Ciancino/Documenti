\date{}
\usepackage[latin9]{inputenc}
\usepackage{graphicx}
\usepackage{fancyhdr}
\usepackage{ucs}
\usepackage{lastpage}
\usepackage{longtable}
\usepackage{multirow}
\usepackage{hyperref}
\hypersetup{
    colorlinks=true,
    citecolor=black,
    filecolor=black,
    linkcolor=black,
    urlcolor=black
}
\pagestyle{fancy}
\fancyhead{}
\fancyfoot{}
\newcommand{\HRule}{\rule{\linewidth}{0.5mm}}
\rhead{\doctitle \versiondoc - \Ciancino}
\lhead{\setlength{\unitlength}{1mm}
	\begin{picture}(0,0)
		\put(5,0){\includegraphics[scale=0.1]{img/LogoTeam.jpg}}
	\end{picture}}
\cfoot{\thepage\ di \pageref{LastPage}}

%%Comandi particolari per i nomi delle figure. Chiamare come una funzione LaTeX
\newcommand{\Ciancino}{\emph{Ciancino }}
\newcommand{\respProg}{\emph{Responsabile di Progetto }}
\newcommand{\ammProg}{\emph{Amministratore di Progetto }}
\newcommand{\analProg}{\emph{Analista }}
\newcommand{\verifProg}{\emph{Verificatore }}
\newcommand{\programProg}{\emph{Programmatore }}
\newcommand{\progetProg}{\emph{Progettista }}
\newcommand{\proponProg}{\emph{Proponente }}
\newcommand{\commitProg}{\emph{Committente }}
\newcommand{\commProg}{\emph{Prof. Filippo Giraldo}}
\newcommand{\nameproject}{\emph{Progetto RistoFast }}

%%DA QUI IN POI POTETE MODIFICARE

%% INSERIRE QUI IL NOME DEL DOCUMENTO (INSERITE SEMPRE UNO SPAZIO ALLA FINE DEL NOME)
\newcommand{\doctitle}{Progetto RistoFast }

%% INSERIRE QUI LA VERSIONE ATTUALE DEL DOCUMENTO (INSERITE SEMPRE UNO SPAZIO ALLA FINE DELLA VERSIONE)
\newcommand{\versiondoc}{V1.0 }

%%INSERITE QUI LA DATA DI COMPILAZIONE FINALE DEL DOCUMENTO
\newcommand{\datared}{11/09/2012}

%%INSERIRE QUI IL/I REDATTORI
\newcommand{\redattore}{\begin{itemize}
\item Giacomo Bain
\item Marco Begolo
\item Massimo Dalla Pietà
\item Gabriele Facchin
\item Massimiliano Facciolo
\item Jessica Gavagnin
\item Pardis Zohouri H
\end{itemize}}


%%INSERIRE LA TIPOLOGIA DI USO DEL DOCUMENTO [Interno/Esterno]
\newcommand{\usodoc}{Esterno}

%%INSERIRE LA LISTA DI DISTRIBUZIONE DEL DOCUMENTO
\newcommand{\listadistr}{\begin{itemize}
\item \Ciancino
\item \commProg
\end{itemize}}

%%INSERIRE IL SOMMARIO DEL DOCUMENTO
\newcommand{\testosommario}{Questo documento descrive il risultato dell'analisi di mercato, i Business Model relativa al progetto \nameproject redatto dal gruppo \Ciancino per il corso di Gestione di Progetti.}

